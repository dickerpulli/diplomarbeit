\chapter{State of the Art}
\label{cha:sota}
In den letzten Jahren wurde das Thema \textit{Angriffe auf Netzwerke und deren Bek�mpfung} immer intensiver in der Wissenschaft erforscht. Bereits im Jahr 1990 wurde von Heady et al.\cite{heady} die Erkennung von Anomalien in Netzwerken behandelt. So wurden in den folgenden Jahren viele Systeme entwickelt, die sich mit dieser Anomalieerkennung besch�ftigen, wie beispielsweise die LOF-Methode (\textit{Local Outlier Factor}) von Breuning et al.\cite{lof}. Aber auch andere Methoden wurden entwickelt. Damit stieg die Notwendigkeit des Vergleiches, so dass von Lazarevic et al.\cite{lazarevic} verschiedene Methoden zur Anaomalieerkennung getestet und miteinander verglichen werden. Die erw�hnte LOF-Methode wird dabei als die zuverl�ssigste bewertet.

Neben der Anomalieerkennung sind weitere Methoden zum Schutz von Netzwerken entworfen worden. Verschiedenste Filter wurden entwickelt um \textit{Distrisbuted Denial-of-Service} (DDoS: siehe Abschnitt~\ref{sec:ddos}) einzud�mmen. So haben Keromytis et al.\cite{sos} eine Methodik zur Verminderung von DoS entwickelt, welche nur authentifizierten Paketen erlaubt ein bestimmtes Ziel zu erreichen. So ein Overlay-basiertes Filtern wurde auch von Weiteren\cite{doslimiting,mayday} untersucht. Weiterhin wurden Filtermethoden vorgeschlagen, die das \textit{Spoofen} von IP-Adressen verhindern sollen\cite{filtersource} oder durch sogenanntes \textit{traceback}, d.h. durch eine schrittweise R�ckverfolgung, die Quellen solcher Attacken herausfiltern\cite{iptraceback,iptraceback2}. Eine dritte Methode, das \textit{pushback}\cite{pushback,pushback2}, erm�glicht das Platzieren von Netzwerkfiltern in Netzwerken, die der Quelle der Attacke n�her sind.

Im Bereich der Verhinderung bzw. Abschw�chung von Angriffen auf gro�e Netzwerke wurde schon viel geforscht. Die Forschung ist bisher jedoch noch auf der Suche nach dem Optimum. Bisher wurde noch keine Methode gefunden beispielsweise DDoS wirksam zu bek�mpfen.

Die Entwicklung von Gegenma�nahmen gegen Angriffe auf Netzwerke bzw. einzelne Teilnehmer von Netzwerken ben�tigt zun�chst theoretische Kenntnisse. Dazu geh�ren Wissen �ber Aufbau von Netzwerken. Zudem auch die Kenntnisse �ber die Methoden von Angriffen. Diese Theorie wird im folgenden Kapitel er�rtert.
