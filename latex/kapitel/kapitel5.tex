\chapter{Fazit}
\label{cha:fazit}
Aus dem Diagramm (Abbildung~\ref{fig:diagramm}) ist die Wirksamkeit des Whitelist-Ansatzes zu erkennen. Mit steigender Versuchsdauer erh�ht sich die Wahrscheinlichkeit eines erfolgreichen Verbindungsaufbaus, deren Wirkung jedoch nur sehr gering ist. Es ist zu erwarten, dass in einem l�nger andauernden Szenario, die Kurve "`Mit DDoS und Whitelist"' noch weiter an Steigung zunimmt. Dies liegt in der Aufnahme von weiteren Web-Clients in die Whitelist begr�ndet.

W�hrend eines stattfindenden SYN-Flood ist die Aufnahme in die Whitelist nur schwer m�glich. Ein Timeout verkleinert die Liste der halboffenen Verbindungen um jeweils ein Element. Nur in diesem Fall besteht M�glichkeit, dass eine Anfrage nicht verworfen wird. Da neben den Anfragen der Web-Clients aber weiterhin noch SYN-Pakete von Bots gesendet werden, ist die Wahrscheinlichkeit sehr gering, dass der Anfrage des Web-Clients genau dieser frei gewordene Platz zugeteilt wird. Nur wenn das SYN-Paket des legitimen Web-Clients nicht verworfen wird, besteht die M�glichkeit, dass die Verbindung zustande kommt. Einzig in diesem Fall wird der Web-Client in die Whitelist aufgenommen.

In diesem Punkt zeigt das Whitelist-Prinzip einen Schwachpunkt. Der Angreifer k�nnte beispielsweise die Methodik der Aufnahme in die dynamische Whitelist erraten oder durch Probieren heraus finden.

Bei dem hier implementierten Ansatz erfolgt die Aufnahme nach dem Senden eines ACK-Pakets. Die Bots des Angreifers k�nnten ein solches Paket aber auch senden. Damit w�rden sie in die Whitelist aufgenommen. Dies h�tte die Unwirksamkeit der Whitelist zur Folge.

Diese Whitelist-Methode ist nur eine von vielen Methoden zur Abwehr von Angriffen bzw. zur Begrenzung von Sch�den durch Angriffe. Mit Hilfe von NeSSi k�nnten sich weitere Strategien entwickeln lassen. Das Einbinden des IRC-Protokolls als Kommunikationsmedium f�r Angreifer und Bots hat dem Simulator zus�tzliche Flexibilit�t verliehen. Auf dieser Grundlage best�nde die M�glichkeit, weitere Gegenma�nahmen zur DDoS-Abwehr zu entwickeln und weitere Angriffsszenarien zu implementieren. Die implementierte Kommunikation �ber IRC er�ffnet Perspektiven zuk�nftiger Forschungsanstrengungen.
